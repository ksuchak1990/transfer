\section{Thesis Structure}\label{sec:structure}

%Ideas for subsequent research:
%\begin{itemize}
    %\item Further exploration of EnKF:
    %\begin{itemize}
        %\item Varying levels of data coverage:
        %\begin{itemize}
            %\item Information on all agents vs some agents
            %\item Aggregated sensor data vs individual coordinate data
            %\item Varying level of agent attribute knowledge, e.g. do we know
                %about agents' destination? If not do we attempt to fix this? If
                %yes then how?
        %\end{itemize}
        %\item How does effectiveness vary with the level of non-linearity in the
            %model?
        %\item How do we measure the level of non-linearity?
    %\end{itemize}
    %\item Comparison of EnKF with other DA methods given the same model:
    %\begin{itemize}
        %\item PF, UKF
        %\item Compare how effective they are
        %\item Compare time complexity
        %\item Compare space complexity
    %\end{itemize}
    %\item Application to a confined case study using real data:
    %\begin{itemize}
        %\item Headrow gateway pedestrian surveys
        %\item Development of city square:
        %\begin{itemize}
            %\item Planning of sensor network deployment
            %\item Use of DA both before and after developments
            %\item What would be the point here?
        %\end{itemize}
    %\end{itemize}
%\end{itemize}

This section proposes a structures for the final output of the PhD.
At present, this is expected to take the traditional form of a thesis, the
structure of which is outlined in Section \ref{sub:structure:thesis}; the
alternative option of pursuing a PhD by publication is also considered in
Section \ref{sub:structure:publication}.
The section also addresses possibilities for other research and outputs in the
form of case studies in conjunction with Leeds City Council in Section
\ref{sub:structure:case_studies}.

\subsection{PhD by Thesis}\label{sub:structure:thesis}

Given the aims and objectives outline in Section \ref{sub:intro:aims}, the
following structure is proposed for the thesis.

\paragraph{Introduction}

This section will be based on Section \ref{sec:intro}, providing the background
and rationale for the project, as well as outlining the main aims and objectives
of the investigation.


\paragraph{Literature Review}

The foundations of this section will likely mirror Section \ref{sec:lit_rev},
along with the addition of some coverage of:
\begin{itemize}
    \item The use of data assimilation schemes in conjunction with
        agent-based models in contexts other than social simulation,
    \item The use of methods other than data assimilation for real-time
        pedestrian modelling (as well as a comparison between such
        approaches and data assimilation methods),
\end{itemize}

\paragraph{Methodology}

This will be based on Section \ref{sec:method}, seeking to outline data
assimilation and how it works, and will be expanded to include any other data
assimilation schemes that I go on to use.

\paragraph{Exploring the Ensemble Kalman Filter with a simple ABM}

This section will be based upon the work presented in Section
\ref{sec:research}, expanding upon it by further exploring the impact of varying
levels of data coverage (knowledge of all agent locations vs knowledge of a
subset of agent location) and varying levels of data aggregation (knowledge of
individual agent locations vs knowledge of number of agents in a given area).
Furthermore, the research currently being undertaken in Section
\ref{sec:research} assumes that the data assimilation method has knowledge
regarding the origin and destination of each agent in the system; this is an
unrealistic assumption, and so an investigation into the impact removing such
knowledge on filter performance would also be involved. 
This part of the investigation would seek to understand how the different
factors impact the performance of the Ensemble Kalman Filter.
Furthermore, the hope is that this would act as a proof-of-concept.

\paragraph{Comparison of the Ensemble Kalman Filter with Different Models}

The research undertaken thus far has focused on implementing the Ensemble Kalman
Filter for a relatively simple agent-based model --- in many of the scenarios
for this model, agent motion is linear and deterministic.
There are likely many scenarios for which this is not the case.
This section would therefore seek to apply the Ensemble Kalman Filter to
different models of pedestrian motion with a view to understanding what facets
of pedestrian motion the method struggles to capture.

\paragraph{Comparison of Different Data Assimilation Methods}

At present, this work has focused on the use of the Ensemble Kalman Filter in
conjunction with agent-based models.
There exist, however, a number of other data assimilation methods --- some of
which are being actively applied to the same model.
This part of the investigation would seek to compare the different methods with
regards to effectiveness in improving simulation accuracy, time complexity and
space complexity.

\paragraph{Conclusion}

This section will draw together the research results, discussing them and how
they pertain to the aims and objectives and the literature review.

\subsection{PhD by Publication}\label{sub:structure:publication}

The structure outlined above pertains to the traditional format of `PhD by
Thesis'.
An alternative to this would be to pursue the `PhD by Publication' route, in
which the final submission would comprise of a series of papers along with and
introductory section and a conclusion section.
The benefits of such an approach would be that it would require that less time
was set aside at the end of the PhD dedicated solely to writing the thesis, and
ensuring that research is published and disseminated early in the career.
The risk of this approach is that it requires that the student has one paper
published, one in review and one ready to submit by the end of the PhD ---
papers should therefore be submitted well in advance to account for time taken
on corrections and alterations.
If this research were to follow such an approach, the following section may be
candidates for publication:
\begin{itemize}
    \item Exploring the Ensemble Kalman Filter with a simple ABM
    \item Comparison of the Ensemble Kalman Filter with Different Models
    \item Comparison of Different Data Assimilation Methods
\end{itemize}

The section of the methodology pertaining to the Ensemble Kalman Filter would be
covered in the first publication, with any subsequent data assimilation methods
being covered in the last publication.

%Given the work undertaken thus far (as outlined in Section \ref{sec:results}),
%there are a number of avenues along which future work may proceed.
%The first and most pressing avenue would be to further explore and develop the
%implementation of the Ensemble Kalman Filter for Agent-Based Models.
%Beyond this, in light of the parallel development of other data assimilation
%methods for agent-based models, a comparison of the Ensemble Kalman Filter with
%other methods should be pursued.
%Finally, the data assimilation scheme may be applied to a case-study scenario in
%order to show it's efficacy in working with real-world environments and data;
%such a piece of work would....
%Each of these avenues will be elaborated upon in the subsequent subsections.

%\subsection{Further Exploration of the Ensemble Kalman
%Filter}\label{sub:timetable:further}

%\subsection{Comparison with other Data Assimilation
%Methods}\label{sub:timetable:comparison}

%\subsection{Application to a Case Study}\label{sub:timetable:application}

\subsection{Case Studies}\label{sub:structure:case_studies}

Beyond the purely academic work outline so far, there is also scope to pursue
external research with Leeds City Council (who are the industrial partner for
this project).
Discussions regarding such case studies are ongoing, and may pertain to either
of the following:
\begin{itemize}
    \item \textbf{Pedestrian Motion on Briggate}: The previous work that has
        been undertaken to apply the Ensemble Kalman Filter to pedestrian
        agent-based models has focused on Briggate. 
    \item \textbf{Renovation of Leeds Station}: The renovation of Leeds Station
        is presently ongoing. There are also plans for the redevelopment of
        parts of Leeds City Square. Members of Leeds City Council are therefore
        interested in understanding the impact of such changes, and exploring
        how the different potential layouts of Leeds City Square will affect the
        flow of pedestrians.
    \item \textbf{Redevelopment of the Headrow}: The redevelopment of the
        Headrow has recently begun whereby Leeds City Council seek to remove the
        central partition in the road with a view to easing the flow of public
        transport, better providing for cyclists, and widening pedestrian
        walkways and offering more green-space.
\end{itemize}

\newpage
