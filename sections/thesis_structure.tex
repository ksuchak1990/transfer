\section{Thesis Structure}\label{sec:structure}

Ideas for subsequent research:
\begin{itemize}
    \item Further exploration of EnKF:
    \begin{itemize}
        \item Varying levels of data coverage:
        \begin{itemize}
            \item Information on all agents vs some agents
            \item Aggregated sensor data vs individual coordinate data
            \item Varying level of agent attribute knowledge, e.g. do we know
                about agents' destination? If not do we attempt to fix this? If
                yes then how?
        \end{itemize}
        \item How does effectiveness vary with the level of non-linearity in the
            model?
        \item How do we measure the level of non-linearity?
    \end{itemize}
    \item Comparison of EnKF with other DA methods given the same model:
    \begin{itemize}
        \item PF, UKF
        \item Compare how effective they are
        \item Compare time complexity
        \item Compare space complexity
    \end{itemize}
    \item Application to a confined case study using real data:
    \begin{itemize}
        \item Headrow gateway pedestrian surveys
        \item Development of city square:
        \begin{itemize}
            \item Planning of sensor network deployment
            \item Use of DA both before and after developments
            \item What would be the point here?
        \end{itemize}
    \end{itemize}
\end{itemize}

Given the work undertaken thus far (as outlined in Section \ref{sec:results}),
there are a number of avenues along which future work may proceed.
The first and most pressing avenue would be to further explore and develop the
implementation of the Ensemble Kalman Filter for Agent-Based Models.
Beyond this, in light of the parallel development of other data assimilation
methods for agent-based models, a comparison of the Ensemble Kalman Filter with
other methods should be pursued.
Finally, the data assimilation scheme may be applied to a case-study scenario in
order to show it's efficacy in working with real-world environments and data;
such a piece of work would....
Each of these avenues will be elaborated upon in the subsequent subsections.

\subsection{Further Exploration of the Ensemble Kalman
Filter}\label{sub:timetable:further}

\subsection{Comparison with other Data Assimilation
Methods}\label{sub:timetable:comparison}

\subsection{Application to a Case Study}\label{sub:timetable:application}

