\section{Results}\label{sec:results}

Experiments:
\begin{itemize}
    \item Varying assimilation period and ensemble size, constant population
    \item Varying assimilation period and population, constant ensemble size
    \item Varying ensemble size and population, constant assimilation period
\end{itemize}

Visualisations:
\begin{itemize}
    \item Heat map of errors for ap vs es, ap vs pop, es vs pop
    \item Snapshot of model with ensemble members linked to relevant states
\end{itemize}

As outlined in Chapter \ref{sec:method}, the Ensemble Kalman Filter is a data
assimilation method which aims to approximate the effect of the original Kalman
Filter by representing the state of the model on which it is operating as an
ensemble of state samples.
The aim of applying this ensemble method is to overcome the difficulties
encountered by the original Kalman Filter when being applied to non-linear
models, and when attempting to propagate the state covariance matrix for models
with high dimensionality.
This method has been applied to an agent based model of pedestrian movement
(documented in Section \ref{sub:method:model}) with a view to reducing the error
in the model with respect to the ground truth.
This Section will therefore aim to present the results of the preliminary
experiments undertaken.
This is be achieved by first showing the impact of the filtering process on the
overall model state, then focusing in on a single agent in the model.
Finally a comparison of the model error for before updating, after updating and
the observations used to update will be provided at each of the time-steps in
which assimilation takes place.
It should be noted that this section adopts the common terminology of
``forecast'' to mean the state predicted by the model before updating, and
``analysis'' to mean the state after updating.

