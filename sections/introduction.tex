\section{Introduction}\label{sec:intro}

%\begin{itemize}
    %\item What am I doing?
    %\item Why am I doing it?
    %\item How have people done it in the past?
    %\item How do I plan to do it?
%\end{itemize}

A better understanding of how people move around their environment is of great
utility to both academics and policy-makers.
Such knowledge can be made use of in the contexts of urban planning, event
management and emergency response, particularly when considering urban
environments.
Furthermore, this may also be of use to those interested in the social issues of
mobility, inclusivity and accessibility of opportunities.

When considering such concepts, investigators often make use of modelling
techniques.
At their most fundamental, models represent our understanding of
the system that we are studying --- an understanding that may not be perfect
\citep{stanislaw1986tests}.
There exist modelling techniques for the simulation of how pedestrians move
around urban spaces.
However, these methods exist largely in isolation of the real-world --- that is
to say that whilst the simulations aim to reflect the real-world, there is no
method by which we can incorporate up-to-date observations into these models to
stop their divergence from reality.

Simulating pedestrian behaviour is often undertaken at the micro-scale, with
such models typically aiming to model at the individual level or on a spatially
fine-grained grid \citep{burstedde2001simulation}.
One of the most prevalent simulation methods in this field is that of
Agent-Based Modelling.
Such methods consist of two key components: agents and environments.
In an Agent-Based Model, we prescribe sets of rules by which individuals interact with each
other and their local environments; as interactions take place on the
micro-scale, we typically observe the emergence of structure at the macro-scale
such as crowding \citep{batty2003discrete} or lane formation \citep{liu2014agent}.
The evaluation of these rules is often not deterministic and instead
introduces some element of randomness; these stochastic elements aim to emulate
the variability of human behaviour.
The introduction of such randomness in conjunction with an imperfect
understanding of the phenomena at play, however, typically result in simulation
runs diverging from the real system.

In constructing their models, agent-based modellers undertake a development
process that involves model verification, validation and calibration.
We can take these to mean the following:
\begin{itemize}
    \item \textbf{Model verification}: the process of ensuring that the implementation is
        an accurate representation of the model \citep{xiang2005verification}.
    \item \textbf{Model validation}: the process of ensuring that the chosen model is an
        accurate representation of the phenomenon that we wish to study
        \citep{crooks2008key}.
    \item \textbf{Model calibration}: the process of searching for model parameter values
        such that we can achieve validation \citep{thiele2014facilitating}.
\end{itemize}
Beyond this, modellers also make efforts to ensure that the initial model
conditions are realistic by setting them based on historical data.

The practices of validation, calibration and setting initial model states based
on historical data are appropriate for offline evaluations such as testing designs
of new buildings or experimenting with different individual behaviours; however,
when aiming to simulate events in real-time, this simply delays the inevitable
divergence of the model from the real system.
Furthermore, model parameters may be transient and thus require to be updated as
time passes and the dynamics evolve.

Given the apparently inevitable divergence of stochastic simulations from the
real systems that they aim to model, one may alternatively turn to big data.
Data is now being generated in higher volumes and at greater velocity than ever
before \citep{chen2014big}; however, there also exist issues with observation
data from such systems.
Whilst models typically allow us to simulate a whole system, observations are
typically sparse in either time or space (or both); this is to say that
observations rarely provide complete coverage of the events.
We therefore seek a solution whereby we can integrate up-to-date observations
into our models as the models continue to simulate the system.

One of the methods by which we can combine knowledge represented by our model
with observations as they become available is through data assimilation
techniques, which are most commonly used in the field of numerical weather
prediction \citep{kalnay2003atmospheric}.
Such techniques are typically made up of two steps:
\begin{enumerate}
    \item \textbf{Predict}: Run the model forward, estimating the state of the system,
    until new observations become available.
    \item \textbf{Update}: Upon receipt of new observations, combine the model's estimate
    of the system state with the new data.
\end{enumerate}
These steps are repeated iteratively in a cycle.
It is important to note that just as the there is error associated with the
model, we also acknowledge that there is observational error associated with the
data.
The aim of incorporating the observations into the model is to improve the model
accuracy with respect to the true system state.

A large volume of work exists in which such techniques are applied to
meteorological systems where the models used are based on differential
equations.
Significantly less work exists in which data assimilation methods are applied to
agent-based models --- in particular pedestrian models.
This dissertation therefore aims to expand on the pre-existing work by
implementing a data assimilation scheme known as the Ensemble Kalman Filter in
conjunction with a relatively simple agent-based model of pedestrians crossing a
two-dimensional station from one side to the other.

\subsection{Aims and Objectives}\label{sub:intro:aims}

With the above in mind, this PhD research project aims to assess whether the
Ensemble Kalman Filter method of data assimilation can be used to improve the
accuracy with which an agent-based model can simulate pedestrian movements
within urban environments.
This will contribute to a young and emerging area of research which seeks to
model social systems at close to real-time.
Ultimately, the hope is that success in this research will feed into the work
undertaken at Leeds City Council (with whom this project is partnered) to help
with data-driven decision making processes.
In order to achieve this, the following research objectives are set out:
\begin{enumerate}
    \item Develop an implementation of the Ensemble Kalman Filter for
        implementation with agent-based models of pedestrian movement.
    \item Evaluate the impact of filter parameters on the data assimilation
        method when implemented with an agent-based model.
    \item Evaluate the impact of varying levels of observational information
        being provided to the filtering technique.
    \item Explore the how varying model and agent behaviours impact filter
        performance.
    \item Compare the performance of the Ensemble Kalman Filter with other
        techniques for updating agent-based models of pedestrian motion.
\end{enumerate}

The remainder of this report will provide the basis for approaching the above
objectives.
This includes outlining a proposal for the structure of the final PhD Thesis in
Section \ref{sec:structure}, reviewing existing literature in the research field
in Section \ref{sec:lit_rev} and providing an overview of the data assimilation
method around which this work centres --- the Ensemble Kalman Filter --- in
Section \ref{sec:method}.
The report will then go on to cover the research that has been performed so far
in Section \ref{sec:research}, before closing with a timetable for the remaining
two years of the PhD and a training review in Section \ref{sec:timetable} and
\ref{sec:training} respectively.

%In order to achieve this, the following objectives are set out:
%\begin{enumerate}
    %\item Develop a general Ensemble Kalman Filter computational class.
    %\item Apply the Ensemble Kalman Filter to the pedestrian model.
    %\item Compare the accuracy with which the Ensemble Kalman Filter
        %implementation of the model simulates the pedestrian movement against
        %the based model without data assimilation.
%\end{enumerate}

%The remainder of this dissertation will provide an overview of some of the work
%that has been undertaken thus far on implementing data assimilation schemes with
%agent based models (Chapter \ref{sec:lit_rev}), elaborate on how data
%assimilation works --- particularly focussing on the aforementioned Ensemble Kalman
%Filter (Chapter \ref{sec:method}) --- and examine the effectiveness of this
%method in improving the accuracy of the agent-based model (Chapter
%\ref{sec:research}).
